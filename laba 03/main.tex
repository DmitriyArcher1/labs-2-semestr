\documentclass{article}
\usepackage{gensymb}
\usepackage{amsfonts}
\usepackage{amsmath}
\usepackage{amssymb}
\usepackage{ragged2e}
\usepackage[utf8]{inputenc}
\usepackage[russian]{babel}

\begin{document}

% -- Первая страница --
\noindent  ОБЩИЕ ЛИНЕЙНЫЕ УРАВНЕНИЯ ГИПЕРБОЛИЧЕСКОГО ТИПА
\newline и аналогично

\begin{equation*}
\int\limits_{M}^{P} (H \, d\eta - K \, d\xi) = - (uv)_M + (uv)_P + \int\limits_{M}^{P} \left( 2 \frac{\partial v}{\partial s} - \frac{b - a}{\sqrt{2}} v \right)u ds .
\end{equation*}

\noindent Отсюда из формулы (6) следует:

\begin{equation*}
    (uv)_M = \frac{(uv)_P + (uv)_Q}{2} + \int\limits_{P}^{M}
    \left(\frac{\partial v}{\partial s} - \frac{b - a}{2\sqrt{2}}v \right)uds + \int\limits_{Q}^{M} \left(\frac{\partial v}{\partial s} - \frac{a + b}{2\sqrt{2}}v \right)u ds +
\end{equation*}

\begin{equation*}
    + \frac{1}{2} \int\limits_{P}^{Q} (H \, d\eta - K \, d\xi) - \frac{1}{2} \iint\limits_{MPQ} (v \mathcal{L}[u] - u \mathcal{M}[v] \, d\xi \, d\eta . 
    \begin{flushright}
        (8)
    \end{flushright}
\end{equation*}

\newline Эта формула является тождеством, верным для любых достаточно гладких функций $u$ и $v$

\newline Пусть $u$ - решение поставленной выше задачи с начальными условиями, а функциия $v$ зависит от точки  $M$ как от параметра и удовлетворяет следующим требованиям:

\begin{equation*}
    \mathcal{M}[v] = v_{\xi \xi} - v_{\eta \eta} - (av)_{\xi} - (bv)_{\eta} + cv = 0 \quad \text{внутри} \triangle MPQ \quad \begin{flushright}
        (9)
    \end{flushright}
\end{equation*}

\noindent и

\begin{align*}
\begin{cases}
    & \frac{\partial v}{\partial s} = \frac{b - a}{2\sqrt{2}} v \quad \text{на характеристике } MP, \\
    & \frac{\partial v}{\partial s} = \frac{b + a}{2\sqrt{2}} v \quad \text{на характеристике } MQ,
\end{cases}
\end{align*}

\begin{center}
v(M) = 1.    
\end{center}





\noindent Из условий на характеристиках и последнего условия находим:

\begin{align*}
    v = e^{\int\limits_{s_0}^{s} \frac{b - a}{2\sqrt{2}} ds} \quad \text{на } MP, \\
    v = e^{\int\limits_{s_0}^{s} \frac{b + a}{2\sqrt{2}} ds} \quad \text{на } MQ, \\
\end{align*}

\noindent где $s_0$ -- значение $s$ в точке $M$. Как мы видели в $\S$ 4, уравнение (9) и значение функции $v$ на характеристиках $MP$ и $MQ$ польностью определяют её в области $MPQ$. Функцию $v$ часто называют функцией Римана.

\newpage

% -- Вторая страница --

\newline Таким образом, формула (8) для функции $u$, удовлетворяющей уравнению (7), принимает следующий окончательный вид:

\begin{equation*}
\begin{split}
    u(M) &= \frac{(uv)_P + (uv)_Q}{2} + \frac{1}{2} \int\limits_{P}^{Q} \left[ v(u_{\xi} d\eta + u_{\eta} d\xi) - u(v_{\xi} d\eta + v_{\eta} d\xi) + \right. \\
    &+ \left. uv (a \, d\eta - b \, d\xi) \right] + \frac{1}{2} \iint\limits_{MPQ} v(M, M') f(M') d\sigma_{M'} \quad (d\sigma_{M'} = d\xi \, d\eta).
\end{split}
\begin{flushright}
        (10)
\end{flushright}
\end{equation*}

\noindent Эта формула решает поставленную задачу, так как выражения стоящие под знаком интеграла вдоль $PQ$, содержат функции, известные на дуге $C$. В самом деле, функция $v$ была определена выше, а функции

\begin{align*}
    u|_{C} &= \varphi(x), \\
    u_x|_{C} &= u_s \cos(x, s) + u_n \cos(x, n) = \frac{\varphi'(x) - \psi(x) f'(x)}{\sqrt{1 + (f'(x))^2}}, \\
    u_y|_{C} &= u_s \cos(y, s) + u_n \cos(y, n) = \frac{\varphi'(x)f'(x) + \psi(x)}{\sqrt{1 + (f'(x))^2}},
\end{align*}

\noindent вычисляются при помощи начальных данных.

\newline Формула (10) показывает, что если начальные данные известны на дуге $PQ$, то они полностью определеяют функцию в характеристическом $\triangle PMQ$, если функция $f(x, y)$ известна в этой области$^1$).

\newline Формула (10), полученная в предположении существования решения, определяет его через начальные данные и правую часть уравнения (7) и тем самым по существу доказывает единственность решения (ср. с формулой  Даламбера, гл. II, $\S$ 2, стр.51).

\newline Можно показать, что функция $u$, определяемая формулой (10), удовлетворяет условиям задачи (7)-(7'). Однако мы на этом доказательстве не останавливаемся.

\textbf{3. Физическая интерпретация функции Римана.} \noindent Выясним физический смысл функции $v(M, M').$ Для этого найдём решение неоднородного уравнения

\begin{equation*}
    \mathcal{L}[u] = -2f_1 \quad (f = 2f_1)
\end{equation*}

\noindent с нулевыми начальными условиями на кривой С. Обращаясь к формуле (10), видим, что искомое решение имеет вид

\begin{equation*}
    u(M) = \iint\limits_{MPQ} v(M, M') f_1(M') d\sigma_{M'}. 
    \begin{flushright} 
        (11)
    \end{flushright}
\end{equation*}
 

\end{document}
